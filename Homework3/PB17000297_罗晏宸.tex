\documentclass{article}
\usepackage[UTF8]{ctex}
\usepackage[T1]{fontenc}
\usepackage[utf8]{inputenc}
\usepackage{latexsym}
\usepackage{amsmath}
\usepackage{geometry}
\usepackage{ulem}
\usepackage{listings}
\usepackage{xcolor}
\lstset{
    basicstyle = \ttfamily,
    keywordstyle = \bfseries \color{blue!70},
    linewidth = \linewidth,
    xleftmargin = .1\textwidth,
    xrightmargin = .1\textwidth,
    numbers = left,
    numberstyle = \tiny,
    commentstyle = \color{red!50!green!50!blue!50},
    frame = shadowbox,
    rulesepcolor = \color{red!20!green!20!blue!20},
    breaklines = true,
    breakatwhitespace = true,
}

\geometry{left = 2cm, right = 2cm}
\makeatletter
\renewcommand{\section}{\@startsection{section}{1}{0mm}
                                {-5ex plus -.5ex minus -.2ex}
                                {3ex plus .2ex}
                                {\normalfont\large\bfseries}}
\makeatother
\title{Homework 3}
\author{PB17000297 罗晏宸}
\date{March 25 2020}

\begin{document}
\maketitle

\section{给定下面的基本表:
  \protect\\ \indent 学生表 \texttt{Student(\uline{sno}, sname, age)}
  \protect\\ \indent 课程表 \texttt{Course(\uline{cno}, cname, type, credit)}
  \protect\\ \indent 选课表 \texttt{SC(\uline{sno}, \uline{cno}, score, \uline{term})}
  \protect\\ 其中:
  \protect\\ \indent \texttt{type}是整型,\texttt{0}表示必修课,\texttt{1}表示选修课,\texttt{2}表示通识课,\texttt{3}表示公选课。
  \protect\\ \indent \texttt{credit}表示课程学分。
  \protect\\ \indent \texttt{term}表示第几学期,取值范围为\texttt{1-8}。
  \protect\\ 请用SQL语句回答下面的查询:}
\subparagraph{(1)} 查询选修了必修课但是缺少成绩的学生学号和姓名
\subparagraph{(2)} 查询已选必修课总学分大于16并且所选通识课成绩都大于75分的学生姓名
\subparagraph{(3)} 查询所有课程总成绩排名在前50\%(向上取整)的学生中必修课平均分最高的前10位同学,要求返回这些学生的学号、姓名、必修课平均分以及课程总成绩(不足10位时则全部返回)
\subparagraph{(4)} 查询每门课程的课程名、课程类型、平均成绩和不及格率,要求结果按通识课、必修课、选修课、公选课顺序排列(提示:课程名可能有重名)
\subparagraph{(5)} \texttt{SC}表中重复的\texttt{sno}和\texttt{cno}意味着该学生重修了课程(在不同的学期里),现在我们希望删除学生重复选课的信息,只保留最近一个学期的选课记录以及成绩,请给出相应的SQL语句

\paragraph{解}
\subparagraph{(1)} 查询选修了必修课但是缺少成绩的学生学号和姓名

\begin{lstlisting}[language = SQL]
select sno, sname from Student
where exists
(
    select * from SC natural join Course
    where score is NULL and type = 0
);
\end{lstlisting}

\subparagraph{(2)} 查询已选必修课总学分大于16并且所选通识课成绩都大于75分的学生姓名

\begin{lstlisting}[language = SQL]
select sname from Student
where sno in
(
    select sno from SC natural join Course
    where type = 0
    group by sno
    having sum(credit) > 16
) and not exists
(
    select sno from SC natural join Course
    where type = 2 and score <= 75
);
\end{lstlisting}

\subparagraph{(3)} 查询所有课程总成绩排名在前50\%(向上取整)的学生中必修课平均分最高的前10位同学,要求返回这些学生的学号、姓名、必修课平均分以及课程总成绩(不足10位时则全部返回)

\begin{lstlisting}[language = SQL]
with Total(sno, total_score) as
(
    select sno, sum(score) from SC natural join Course
    group by sno
)
select sno, name, average_score, total_score from
(
    Student natural join
    (
        select sno, avg(score) as average_score from SC natural join Course
        where type = 0
        group by sno
    )
    natural join Total
) as t1 where
(
    select count(*) from Total as t2
    where t2.total_score > t1.total_score
)
< ceil((select count(*) from Student) / 2)
order by average_score desc limit 10;
\end{lstlisting}

\subparagraph{(4)} 查询每门课程的课程名、课程类型、平均成绩和不及格率,要求结果按通识课、必修课、选修课、公选课顺序排列(提示:课程名可能有重名)

\begin{lstlisting}[language = SQL]
with Total(cno, total_number) as
(
    select cno, count(*) from SC
    group by cno
),
Fail(cno, fail_number) as
(
    select cno, count(*) from SC
    where score < 60
    group by cno
)
select cname, type, average_grade, fail_rate from Course natural join
(
    select cno, avg(score) as average_grade from SC
    group by sno
) natural join
(
    select cno, (fail_number / (cast(total_number as float))) as fail_rate from Total natural join Fail
)
order by field(type, 2, 0, 1, 3);
\end{lstlisting}


\subparagraph{(5)} \texttt{SC}表中重复的\texttt{sno}和\texttt{cno}意味着该学生重修了课程(在不同的学期里),现在我们希望删除学生重复选课的信息,只保留最近一个学期的选课记录以及成绩,请给出相应的SQL语句

\begin{lstlisting}[language = SQL]
delete from SC as t1
where t1.term <>
(
    select max(t2.term)
    from SC as t2
    where t1.sno = t2.sno and t1.cno = t2.cno
);
\end{lstlisting}

\end{document}
