\documentclass{article}
\usepackage[UTF8]{ctex}
\usepackage[T1]{fontenc}
\usepackage[utf8]{inputenc}
\usepackage{latexsym}
\usepackage{amsmath}
\usepackage{geometry}
\usepackage{ulem}
\usepackage{listings}
\usepackage{xcolor}
\usepackage{cancel}
\usepackage{amsthm}
\usepackage{enumerate}
\usepackage{enumitem}
\usepackage{amssymb}
\setlist{
    leftmargin = .1\linewidth,
    % rightmargin = .1\linewidth,
    % label=\emph{\alph*}.
}
\lstset{
    basicstyle = \ttfamily,
    keywordstyle = \bfseries \color{blue!70},
    linewidth = \linewidth,
    xleftmargin = .1\textwidth,
    xrightmargin = .1\textwidth,
    numbers = left,
    numberstyle = \tiny,
    commentstyle = \color{red!50!green!50!blue!50},
    frame = shadowbox,
    rulesepcolor = \color{red!20!green!20!blue!20},
    breaklines = true,
    breakatwhitespace = true,
}

\geometry{left = 2cm, right = 2cm}
\makeatletter
\renewcommand{\section}{\@startsection{section}{1}{0mm}
                                {-5ex plus -.5ex minus -.2ex}
                                {3ex plus .2ex}
                                {\normalfont\large\bfseries}}
\makeatother
\title{Homework 4}
\author{PB17000297 罗晏宸}
\date{April 8 2020}

\begin{document}
\maketitle

\section{已知有关系模式 $R(A,\ B,\ C,\ D,\ E)$,$R$ 上的一个函数依赖集如下:$$F = \{ A \rightarrow BC,\ B \rightarrow CE,\ A \rightarrow B,\ AB \rightarrow C,\ AC \rightarrow DE,\ E \rightarrow A \}$$}
\subparagraph{(1)} 求出 $F$ 的最小函数依赖集(要求写出求解过程)
\subparagraph{(2)} 求 $R$ 的候选码,并给出证明

\paragraph{解}
\subparagraph{(1)}
首先利用分解律
\begin{equation*}
    F = \{A \rightarrow B,\ A \rightarrow C,\ B \rightarrow C,\ B \rightarrow E,\ AB \rightarrow C,\ E \rightarrow A,\ AC \rightarrow D,\ AC \rightarrow E\}
\end{equation*}
再消去左部冗余属性
\begin{align*}
    A \rightarrow B,\ A \rightarrow AB,\ AB \rightarrow C & \Rightarrow A \rightarrow C \\
    A \rightarrow C,\ A \rightarrow AC,\ AC \rightarrow D & \Rightarrow A \rightarrow D \\
    A \rightarrow C,\ A \rightarrow AC,\ AC \rightarrow E & \Rightarrow A \rightarrow E
\end{align*}
\begin{equation*}
    F = \{A \rightarrow B,\ A \rightarrow C,\ B \rightarrow C,\ B \rightarrow E,\ A \rightarrow D,\ A \rightarrow E,\ E \rightarrow A\}
\end{equation*}
最后消去冗余依赖
\begin{align*}
    A \rightarrow B,\ B \rightarrow C & \Rightarrow \bcancel{A \rightarrow C} \\
    A \rightarrow B,\ B \rightarrow E & \Rightarrow \bcancel{A \rightarrow E}
\end{align*}
\begin{equation*}
    F = \{A \rightarrow B,\ B \rightarrow C,\ B \rightarrow E,\ A \rightarrow D,\ E \rightarrow A\}
\end{equation*}

\subparagraph{(2)}
$A$和$E$是候选码,证明如下
\begin{proof}
    \begin{align*}
                                                          & A \rightarrow B             \\
        A \rightarrow B,\ B \rightarrow C \Rightarrow     & A \rightarrow C             \\
                                                          & A \rightarrow D             \\
        A \rightarrow B,\ B \rightarrow E \Rightarrow     & A \rightarrow E             \\
                                                          & A \rightarrow ABCDE \in F^+ \\
                                                          &                             \\
        E \rightarrow A,\ A \rightarrow ABCDE \Rightarrow & E \rightarrow ABCDE \in F^+
    \end{align*}
\end{proof}


\section{现有关系模式: $R(A,\ B,\ C,\ D,\ E,\ F,\ G)$,$R$ 上的一个函数依赖集:$$F = \{AB \rightarrow E,\ A \rightarrow B,\ B \rightarrow C,\ C \rightarrow D \}$$}
\subparagraph{(1)} 该关系模式满足第几范式? 为什么?
\subparagraph{(2)} 如果将关系模式 $R$ 分解为: $R_1(A,\ B,\ E)$ ,$R_2(B,\ C,\ D)$,$R_3(A,\ F,\ G)$,该数据库模式最高满足第几范式?
\subparagraph{(3)} 请将关系模式 $R$ 无损连接并且保持函数依赖地分解到 3NF,要求给出具体步骤。
\subparagraph{(4)} 请将关系模式 $R$ 无损连接地分解到 BCNF,要求给出步骤 。

\paragraph{解}
\subparagraph{(1)}
满足第一范式,
\begin{align*}
                                                   & A \rightarrow B \\
    A \rightarrow B,\ B \rightarrow C \Rightarrow  & A \rightarrow C \\
    A \rightarrow C,\ C \rightarrow D \Rightarrow  & A \rightarrow D \\
    A \rightarrow B,\ AB \rightarrow E \Rightarrow & A \rightarrow E
\end{align*}
主码为$A,\ F,\  G$,但有$A \rightarrow B$,因此$R$不是2NF的。

\subparagraph{(2)}
$R_1(A,\ B,\ E)$,$F_1 = \{AB \rightarrow E,\ A \rightarrow B\}$,主码为$A$,满足3NF;
$R_2(B,\ C,\ D)$,$F_2 = \{B \rightarrow C,\ C \rightarrow D\}$,主码为$B$,满足2NF,但存在传递依赖,不满足3NF;
$R_3(A,\ F,\ G)$,$F_3 = \varnothing$,主码为$A,\ F,\  G$,没有非主属性,属满足BCNF;
因此该数据库模式最高满足第二范式。

\subparagraph{(3)}
\begin{enumerate}[label = \emph{\alph*}.]
    \item $R$的最小函数依赖集$F = \{A \rightarrow B,\ B \rightarrow C,\ C \rightarrow D,\ A \rightarrow E\}$
    \item $R'(F,\ G)$,$U = \{A,\ B,\ C,\ D,\ E\}$
    \item $F_1 = \{A \rightarrow B,\ A \rightarrow E\}$,$F_2 = \{B \rightarrow C\}$,$F_3 = \{C \rightarrow D\}$
    \item $q = \{R_1(A,\ B,\ E),\ R_2(B,\ C),\ R_3(C,\ D),\ R_4(F,\ G)\}$
    \item 主码为$A,\ F,\  G$
    \item $p = q \cup R_5(A,\ F,\ G) = \{R_1(A,\ B,\ E),\ R_2(B,\ C),\ R_3(C,\ D),\ R_4(F,\ G),\ R_5(A,\ F,\ G)\}$
\end{enumerate}

\subparagraph{(4)}
\begin{enumerate}[label = \emph{\alph*}.]
    \item $D = \{R\}$,$F = \{A \rightarrow B,\ B \rightarrow C,\ C \rightarrow D,\ A \rightarrow E\}$,候选码为$A,\ F,\  G$
    \item $\{B \rightarrow C\}$不满足BCNF要求,$\{R_1(A,\ B,\ D,\ E,\ F,\ G),\ R_2(B,\ C)\}$
    \item $R_1$主码为$A,\ F,\  G$
    \item $\{B \rightarrow D\}$不满足BCNF要求,$\{R_1(A,\ B,\ E,\ F,\ G),\ R_2(B,\ D),\ R_3(B,\ C)\}$
    \item 满足BCNF要求,结束
\end{enumerate}
\end{document}
