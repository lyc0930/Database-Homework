\documentclass{article}
\usepackage[UTF8]{ctex}
\usepackage[T1]{fontenc}
\usepackage[utf8]{inputenc}
\usepackage[colorlinks, linkcolor = black]{hyperref}
\usepackage{latexsym}
\usepackage{amsmath}
\usepackage{geometry}
\usepackage{ulem}
\usepackage{xcolor}
\usepackage{amsthm}
\usepackage{enumerate}
\usepackage{enumitem}
\usepackage{amssymb}
\usepackage{tikz-er2}
% \usetikzlibrary{er}
\usetikzlibrary{positioning}
\usetikzlibrary{shadows}

\tikzstyle{every entity} = [top color=white, bottom color=blue!30,
                            draw=blue!50!black!100, drop shadow]
\tikzstyle{every weak entity} = [drop shadow={shadow xshift=.7ex,
                                 shadow yshift=-.7ex}]
\tikzstyle{every attribute} = [top color=white, bottom color=yellow!20,
                               draw=yellow, node distance=1cm, drop shadow]
\tikzstyle{every relationship} = [top color=white, bottom color=red!20,
                                  draw=red!50!black!100, drop shadow]
\tikzstyle{every isa} = [top color=white, bottom color=green!20,
                         draw=green!50!black!100, drop shadow]

\setlist{
    leftmargin = .1\linewidth,
    % rightmargin = .1\linewidth,
    % label=\emph{\alph*}.
}

\geometry{left = 2cm, right = 2cm}
\makeatletter
\renewcommand{\section}{\@startsection{section}{1}{0mm}
                                {-5ex plus -.5ex minus -.2ex}
                                {3ex plus .2ex}
                                {\normalfont\large\bfseries}}
\makeatother
\title{HomeworkIn 5}
\author{PB17000297 罗晏宸}
\date{April 22 2020}

\begin{document}
\maketitle

\section{假设某公司要开发一个信息管理系统。根据调研,获得该公司的业务规则如下(所有实体的码可以按日常生活中的一般规则处理):}
\begin{enumerate}[label = (\arabic{*})]
    \item 公司下设几个部门,如技术部、财务部、市场部等;
    \item 每个部门承担多个工程项目,每个工程项目属于一个部门。每个部门为其承担的每个项目分别开设独立的银行账户;
    \item 每个部门有多名职工,每一名职工只能属于一个部门;
    \item 一个职工可能参与多个工程项目,且每个工程项目有多名职工参与施工。根据职工在工程项目中完成的情况发放酬金;
    \item 工程项目有工程号、工程名两个属性;部门有部门号、部门名称两个属性;职工有职工号、姓名、性别属性。
\end{enumerate}
根据以上描述,请画出相应的 ER 模型(使用传统的 ER 建模符号),并将它转换为关系数据库模式。注:关系模式和属性名称均使用中文名称。

\paragraph{解}
ER 模型如图\ref{ER}所示,
\begin{figure}[h]
    \centering
    \begin{tikzpicture}[node distance = 1.5cm, every edge/.style={link}]
        \node[entity] (Department) {部门};
        \node[attribute] (DepartmentName) [above left = of Department] {部门名称} edge (Department);
        \node[attribute] (DepartmentNo) [left = of Department] {\key{部门号}} edge (Department);

        \node[relationship] (UnderTake) [right = of Department] {承办};
        \node[attribute] (Account) [above = of UnderTake] {银行账户} edge (UnderTake);

        \node[entity] (Project) [right = of UnderTake] {工程};
        \node[attribute] (ProjectName) [above right = of Project] {工程名} edge (Project);
        \node[attribute] (ProjectNo) [right = of Project] {\key{工程号}} edge (Project);

        \node[relationship] (WorkIn) [below = of Department] {就职};

        \node[entity] (Staff) [below = of WorkIn] {职工};
        \node[attribute] (StaffName) [below left = of Staff] {姓名} edge (Staff);
        \node[attribute] (StaffNo) [below = of Staff] {\key{职工号}} edge (Staff);
        \node[attribute] (StaffGender) [below right = of Staff] {性别} edge (Staff);

        \node[relationship] (Participate) [below = 4.0cm of Project] {参与施工};
        \node[attribute] (Remuneration) [below right = of Participate] {酬金} edge (Participate);

        \draw[link] (Department.-90) |- node [pos = 0.1, auto] {1} (WorkIn.90);
        \draw[link] (Staff.90) |- node [pos = 0.1, auto, swap] {N} (WorkIn.-90);

        \draw[link] (Department.0) |- node [pos = 0.6, auto] {1} (UnderTake.180);
        \draw[link] (Project.180) |- node [pos = 0.6, auto, swap] {N} (UnderTake.0);

        \draw[link] (Staff.0) |- node [pos = 0.6, auto] {N} (Participate.180);
        \draw[link] (Project.-90) |- node [pos = 0.1, auto, swap] {M} (Participate.90);
    \end{tikzpicture}
    \caption{信息管理系统 ER 模型}
    \label{ER}
\end{figure}
转换为关系数据库模式为
\begin{itemize}
    \item 部门(\uline{部门号}, 名称)
    \item 职工(\uline{职工号}, 姓名, 性别, 部门号)
    \item 工程项目(\uline{工程号}, 工程名, 部门号, 银行账户)
    \item 参与施工(\uline{职工号}, \uline{工程号}, 酬金)
\end{itemize}
\end{document}
