\documentclass{article}
\usepackage[UTF8]{ctex}
\usepackage[T1]{fontenc}
\usepackage[utf8]{inputenc}
\usepackage[colorlinks, linkcolor = black]{hyperref}
\usepackage{latexsym}
\usepackage{amsmath}
\usepackage{geometry}
\usepackage{ulem}
\usepackage{xcolor}
\usepackage{listings}
\usepackage{amsthm}
\usepackage{amssymb}
\usepackage{setspace}
\usepackage{dashrule}
\usepackage{tikz}
\usetikzlibrary{positioning}

\def\rightbrace#1{#1)}

\lstset{
    basicstyle = \ttfamily,
    keywordstyle = \bfseries,
    linewidth = \linewidth,
    xleftmargin=.2\textwidth,
    xrightmargin=.2\textwidth,
    numbers = left,
    numberstyle = \rightbrace,
    frame = none,
    emph = CHECKPOINT,
    emphstyle = \textbf,
    escapeinside = ``,
}

\geometry{left = 2cm, right = 2cm}
\makeatletter
\let\origthelstnumber\lst@PlaceNumber

\renewcommand{\section}{\@startsection{section}{1}{0mm}{-5ex plus -.5ex minus -.2ex}{3ex plus .2ex}
{\normalfont\large\bfseries}}

\newcommand*\Suppressnumber{%
  \lst@AddToHook{OnNewLine}{%
    \let\lst@PlaceNumber\relax%
%    \advance\c@lstnumber-\@ne\relax% Not really necessary
  }%
}

\newcommand\Reactivatenumber[1]{%
  \global\c@lstnumber#1%
  \global\advance\c@lstnumber\m@ne\relax%
  \lst@AddToHook{OnNewLine}{%
  \let\lst@PlaceNumber\origthelstnumber%
  }%
}
\makeatother
\title{Homework 7}
\author{PB17000297 罗晏宸}
\date{May 5 2020}

\begin{document}
\maketitle

\section{什么是事务的 ACID 性质?请给出违背事务 ACID 性质的具体例子,每个性质举一
  个例子。}

\paragraph{解}
\rightbrace{1}
\section{如果一个存储过程 A 的内部调用了另一个存储过程 B,此时 A 和 B 是否都可以使
  用事务编程并保证事务的 ACID 性质?请解释你的理由}

\paragraph{解}

\section{下面是一个数据库系统开始运行后的日志记录,该数据库系统支持检查点。}
设日志修改记录的格式为 \lstinline{<Tid, Variable, Old value, New value>},请给出对于题中所示\textcircled{1}、
\textcircled{2}、\textcircled{3}三种故障情形下,数据库系统恢复的过程以及数据元素 A, B, C, D, E, F 和 G 在执行了恢复过程后的值。
\newpage
\begin{lstlisting}[]
<T1, Begin Transaction>
<T1, A, 10, 40>
<T2, Begin Transaction>
<T1, B, 20, 60>
<T1, A, 40, 75>
<T2, C, 30, 50>
<T2, D, 40, 80>
<T1, Commit Transaction>
<T3, Begin Transaction>
<T3, E, 50, 90>`\tikz[remember picture, baseline] \node [anchor=base, blue] (anchor1) {};`
<T2, D, 80, 65>
<T2, C, 50, 75>
<T2, Commit Transaction>`\tikz[remember picture, baseline] \node [anchor=base, blue] (anchor2) {};`
<T3, Commit Transaction>
<CHECKPOINT>
<T4, Begin Transaction>
<T4, F, 60, 120>
<T4, G, 70, 140>`\tikz[remember picture, baseline] \node [anchor=base, blue] (anchor3) {};`
<T4, F, 120, 240>
<T4, Commit Transaction>
\end{lstlisting}
\begin{tikzpicture}[overlay, remember picture]
    \node (1) [below right = -0.7em and 7em of anchor1] {\textcircled{1}};
    \foreach \i in {2, 3}
        {
            \node (temp\i) [below = -0.6em of anchor\i] {\phantom{\i}};
            \node (\i) at (1 |- temp\i) {\textcircled{\i}};
        }
    \foreach \i in {1, 2, 3}
    \draw[dashed] (\i) +(-1, 0) -- +(-22em, 0);
\end{tikzpicture}
\paragraph{解}

\end{document}
